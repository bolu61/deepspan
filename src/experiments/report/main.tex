% !TeX root = main.tex
% !TeX root = main.tex
\documentclass[sigplan,screen,natbib=false]{acmart}
\usepackage{amsmath}
\usepackage{enumitem}
\usepackage{algorithm}
\usepackage[noend]{algpseudocode}
\usepackage{cleveref}
\usepackage[datamodel=acmdatamodel,style=acmnumeric]{biblatex}
\usepackage{amsmath}
\usepackage{mathtools}
\usepackage{cleveref}

\settopmatter{printacmref=false} % Removes citation information below abstract
\renewcommand\footnotetextcopyrightpermission[1]{} % removes footnote with conference information in the first column
\pagestyle{plain} % removes running headers

\DeclareMathOperator*{\argmax}{arg\,max}
\DeclareMathOperator*{\argmin}{arg\,min}
\DeclarePairedDelimiter{\seq}{\langle}{\rangle}


\title{Project Final Report: Generating Realistic Workloads}
\author{JianChen Zhao}
\email{jianchen.zhao@uwaterloo.ca}

\addbibresource{paperpile.bib}

\begin{document}

\maketitle

\section{Introduction}

In this project, I have set to model the event log generation processes in
large-scale software systems to improve realistic workload synthesis for system
load testing.

\subsection{Background}

Event log-producing workloads can be seen as an interleaved Markov process
(IMP). Such a process can be modeled with an interleaved hidden Markov model
(HMM). It combines a mixture of individual HMM, each modeling a single
independent process. In load testing, tasks can be modeled by such a process. We
can generate new workloads based on the processes using an interleaved HMM
trained on those processes, leading to realistic yet high-variance synthetic
workloads.

\section{Implementing Interleaved HMM}

I have implemented an interleaved HMM using the Flax\cite{Heek2023-nl} framework
and in this section, I will go over the implementation details and how it is
trained.

Consider a single HMM \(\mu\) with states \(1, 2, ...,n\) in \(S_\mu\). To fit
this HMM on a hidden process, we can maximize the likelihood \(\mathcal{L}(\mu |
\seq{y_t}_t) = p_\mu(\seq{y_t}_t)\) of \(\mu\) given observations from the
hidden process \(\seq{y_t}_t = \seq{y_1, y_2, ..., y_t}\). Since an observation
\(y_t\) is only dependant on the current state \(x_t\), we can write this
probability like so:
\begin{equation}\label{eq:likelihood}
    p_\mu(\seq{y_t}_t) = \sum_{x_t} p_\mu(x_t, \seq{y_t}_t),
\end{equation}
where it can be demonstrated by recursively applying the chain rule that for \(1
\leq t \leq T\)
\begin{equation}
    \begin{multlined}
        p_\mu(x_t, \seq{y_t}_t) = \\
        p_\mu(y_t|x_t) \sum_{x_{t-1}} p_\mu(x_t|x_{t-1}) p_\mu(x_{t-1}, \seq{y_t}_{t-1}).
    \end{multlined}
\end{equation}
\begin{equation}\label{eq:max_likelihood}
    p_\mu(x_0, \seq{}) = p_\mu(x_0)
\end{equation}
\(p_\mu(y_t|x_t)\) is the emission probability of \(\mu\), \(p_\mu(x_t | x_t)\)
is the transition probability and \(p_\mu(x_0)\) is the prior or stationary
probability of the states. Then, the fitted model \(\hat{\mu}\) is one that maximizes \cref{eq:likelihood}
\begin{equation}
    \label{eq:argmax_observation}
    \hat{\mu} = \argmax_{\mu} p_\mu(\seq{y_t}_t).
\end{equation}

This recursive equation can be solved using the forward algorithm described in
\cref{alg:forward}. The forward algorithm has space complexity \(O(n)\)  and
time complexity \(O(nT)\) where \(n\) is the number of possible states and \(T\)
is the length of the input sequence. In the context of interleaved HMM \(\mu\),
there are \(K\) individual HMMs \(\mu_k\), the state of \(\mu\) must encode the
states of all \(\mu_k\). Thus, no matter the encoding, the time and space
complexity of the forward algorithm will be exponential with respect to \(K\).
In fact, training the interleaved HMM is NP-hard\cite{Landwehr2008-vw}.
\citeauthor{Landwehr2008-vw} shows an alternative approximate algorithm for
training, but I have yet to implement it.
\begin{algorithm}[H]
    \caption{The forward algorithm.}\label{alg:forward}
    \begin{algorithmic}[1]
        \Function{Forward}{$o_{1..T}$}
        \For{$i \gets 1 .. n$}
        \State $\alpha_i \gets \pi_i$
        \EndFor
        \For{$t \gets 1 .. T$}
        \For{$i \gets 1 .. n$}
        \State $\alpha_i \gets P(o_t|s_{i,t}) \sum_{s_{i, t-1}} P(s_{i, t} | s_{i, t-1}) \alpha_i$Í
        \EndFor
        \EndFor
        \State \Return $\sum_{i} \alpha_i$
        \EndFunction
    \end{algorithmic}
\end{algorithm}

Since the forward algorithm's gradient can be automatically computed by
Jax\cite{Bradbury2018-jz}, on which Flax is based, \cref{eq:argmax_observation}
can be found using gradient descent. Specifically, I used the
Adam\cite{Kingma2014-jj} optimizer on a synthetic dataset described in
\cref{sec:dataset}. I implemented the forward algorithm in \(\log\) space for
better numerical stability and thus the optimized loss is the negative
log-likelihood:
\begin{equation}
    \label{eq:neg_log_likelihood}
    \argmin_{\mu} - \log P(y_{1..T} | \mu).
\end{equation}

\section{Exploring Google Borg Dataset}\label{sec:dataset}

Because the current implementation of the forward algorithm can not handle
complex data using a reasonable amount of resources, I have not trained the
interleaved HMM on the Google Borg dataset. This dataset contains logs generated
from thousands of machines, thus thousands of possible Markov processes.

Instead, I've spent time designing a synthetic dataset using the implemented
interleaved HMM with transition weights sampled from a beta distribution
\(\text{Beta}(\alpha=0.1, \beta=0.9)\).  The number of states and subchains in
the generating model can be customized. This dataset generates interleaved
sequences over an alphabet on the fly. To test my implementation, I trained a
different HMM with randomly initialized logits for the transition and emission
probabilities.

Once trained, I measured the loss of the model, specifically the choice state
variable. I have found that the loss is less than 0.001 which indicates that
either the trained model has learned the original logits of the generating model
close to perfect, or there is a bug which I need to investigate. The possibility
of overfitting is ruled out, since the synthetic dataset is generated on the
fly, and no two examples are the same.

\printbibliography

\end{document}